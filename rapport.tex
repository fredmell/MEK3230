\documentclass[aps,reprint]{revtex4-1}
% Engine-specific settings
% Detect pdftex/xetex/luatex, and load appropriate font packages.
% This is inspired by the approach in the iftex package.
% pdftex:
\ifx\pdfmatch\undefined
\else
    \usepackage[T1]{fontenc}
    \usepackage[utf8]{inputenc}
\fi
% xetex:
\ifx\XeTeXinterchartoks\undefined
\else
    \usepackage{fontspec}
    \defaultfontfeatures{Ligatures=TeX}
\fi
% luatex:
\ifx\directlua\undefined
\else
    \usepackage{fontspec}
\fi
% End engine-specific settings
\usepackage[english]{babel}
\usepackage{csquotes}
% \usepackage[backend=biber, sortcites]{biblatex}
\usepackage{url}
\usepackage{textcomp}
\usepackage[usenames,dvipsnames,svgnames, table]{xcolor}
\usepackage[font={scriptsize}]{caption}
\usepackage{amsmath} \usepackage{amsthm} \usepackage{amsfonts}
\usepackage{amssymb}
\usepackage{enumerate}
\usepackage{tikz} \usepackage{float}
\usepackage[procnames]{listings}
\usepackage{pstool} \usepackage{pgfplots}
\usepackage{wrapfig} \usepackage{graphicx} \usepackage{epstopdf}
\usepackage{afterpage}
\usepackage{physics}
\usepackage{multirow}
\usepackage{gensymb}
\usepackage{algorithm}
\usepackage{microtype}
\usepackage[noend]{algpseudocode}
\usepackage{xcolor,colortbl}
\usepackage{microtype}
\usepackage{geometry}
\usepackage{hyperref}
\usepackage{graphicx}
\usepackage{caption}
\usepackage{subcaption}
\usepackage{lipsum}
% \usepackage{pythontex}
% \usepackage{authblk}
\usepackage{nth}
\usepackage{siunitx}
% \usepackage[toc,page]{appendix}
\floatstyle{plaintop}
\restylefloat{table}

% Custom commands
\newcommand{\unit}[1]{\:\mathrm{#1}}
\newcommand{\noref}[1]{\hyperref[#1]{\ref*{#1}}}
\newcommand{\nonref}[1]{\hyperref[]{\ref*{#1}}}
\newcommand\blankpage{%
  \null
  \thispagestyle{empty}%
  \addtocounter{page}{-1}%
  \newpage}

% Default fixed font does not support bold face
\DeclareFixedFont{\ttb}{T1}{txtt}{bx}{n}{7} % for bold
\DeclareFixedFont{\ttm}{T1}{txtt}{m}{n}{7}  % for normal

\newcommand\numberthis{\addtocounter{equation}{1}\tag{\theequation}}
\DeclareCaptionFont{white}{\color{white}}
\DeclareCaptionFormat{listing}{\colorbox{gray}{\parbox{\columnwidth}{#1#2#3}}}
\pgfplotsset{compat=1.14}


% Biber for references
% \bibliographystyle{aipauth4-1}

\begin{document}
\sisetup{detect-all}
\title{Mimicking the flight of a Dipterocarpus fruit}
\author{Magnus Holm}
\author{Jonas G. S. Lunde}
\author{Frederik J. Mellbye}
\affiliation{University of Oslo, Oslo, Norway}
\date{\today}

\begin{abstract}
Abstract abstract.
\end{abstract}
\maketitle

\section{Introduction}
\label{sec:introduction}
See \cite{instruks} for a full introduction to this experiment. In the experiment
3D models of Dipterocarpaceae fruit were developed in FreeCAD, 3D-printed and
released in a water cylinder. The fruits are equipped with wings, which enables
seed dispersion over vast distances. The seed flight was then assessed with a
variety in important parameters such as fruit mass and wing curvature. Out of the
specific models that were investigated, ideal parameters for maximizing fruit
dispersion were determined.

\section{Theory}
\label{sec:theory}
\subsection{Non-dimensional quantities. Reynolds and Strouhal numbers.}
From the project description, there are nine physical variables which are based
on the dimensions $M$ (mass), $L$ (length) and $T$ (time). By Buckingham's
$\Pi$-theorem, there are
\begin{align*}
  p = n - k = 9 - 3 = 6
\end{align*}
dimensionless groups that can be constructed from the original variables.
Two of the possible $\Pi$-groups are the Reynolds and Strouhal numbers. Using
the nongeometrical parameters these can be written as
\begin{align}
  Re &= \\
  St &=
\end{align}
\subsection{Expression for sinking velocity}

\subsection{Lift and drag forces on a wing element}
By the Kutta-Joukowski condition, for low angles of attack the lift is
proportional to (with inviscid theory)
\begin{align*}
    L \sim \rho U^2 L \sin{\alpha}
\end{align*}
In our case the velocity is given by $U = \omega R$, which yields a lift
\begin{align}
  L \sim \rho \omega^2 R^2 L
\end{align}
This formula might be inaccurate because in the experiment, viscous effects
created a trailing vortex. This is not accounted for by inviscid theory.

From dimensional analysis a possible configuration for the drag force is
\begin{align}
  D \sim \rho R^2 U_V \cos{\phi}
\end{align}
where we have taken the cosine of $\phi$ because if $\phi \rightarrow 0$ the
drag should increase and if $\phi \rightarrow \pi/2$ the drag should decrease. 
\subsection{Moment on the seed}

\section{Experiment method}
\label{sec:method}
\section{Results and discussion}
\label{sec:results}
Ddwadawd
\bibliography{references}
\blankpage
\end{document}

% Local Variables:
% TeX-engine: luatex
% End:
